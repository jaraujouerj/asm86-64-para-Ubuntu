\chapter{Debugger DDD}

\begin{chapquote}{Autor desconhecido}
``Meu software nunca possui bugs. Ele só
desenvolve recursos aleatórios.''
\end{chapquote}

Um depurador permite ao usuário controlar a execução de um programa, examinar variáveis, outra memória (ou seja, espaço na pilha) e exibir a saída do programa (se houver). O GNU Data Display Debugger de código aberto (DDD\footnote{Para mais informações, consulte: https://pt.wikipedia.org/wiki/Data\_Display\_Debugger}) é um front-end visual para o GNU Debugger (GDB\footnote{Para mais informações, consulte: https://pt.wikipedia.org/wiki/GNU\_Debugger}) e está amplamente disponível. Outros depuradores podem ser facilmente usados, se desejado.

Somente os comandos básicos do depurador são abordados neste capítulo. O depurador DDD possui muitos outros recursos e opções não abordados aqui. À medida que você ganha experiência, vale a pena revisar a documentação do DDD, mencionada no Capítulo \ref{cap1}, para aprender mais sobre os recursos adicionais, a fim de ajudar a melhorar a eficiência geral da depuração.

A funcionalidade do DDD pode ser estendida usando vários plug-ins. Os plug-ins não são necessários e não serão abordados neste capítulo.

Este capítulo aborda o uso do depurador GNU DDD como uma ferramenta. O processo lógico de como depurar um programa não é abordado neste capítulo.

\section{Iniciando o DDD}
O depurador \textbf{ddd} é iniciado com o arquivo executável. O programa deve ser montado e vinculado às opções corretas (conforme observado no capítulo anterior). Por exemplo, usando o programa exemplo anterior, por exemplo, o comando seria:
\begin{verbatim}
ddd exemplo
\end{verbatim}

Ao iniciar o DDD/GDB, algo semelhante à tela, mostrada abaixo, deve ser exibida (com o código-fonte apropriado exibido).