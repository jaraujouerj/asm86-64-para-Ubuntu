\chapter{Formato de um programa}

\begin{chapquote}{Autor desconhecido}
``Eu adoraria mudar o mundo, mas eles não vão me dar o código fonte.''
\end{chapquote}
Este capítulo resume os requisitos de formatação para programas em linguagem assembly. Os requisitos de formatação são específicos para o  \textit{assembler} \textbf{yasm}. Outros montadores podem
ser um pouco diferente. Um programa completo na linguagem assembly é apresentado para demonstrar a formatação apropriada do programa.

Um arquivo de origem de montagem formatado corretamente consiste em várias partes principais:
\begin{itemize}
	\item Seção de dados, na qual os dados inicializados são declarados e definidos.
	\item Seção BSS, na qual dados não inicializados são declarados.
	\item Seção de texto, na qual o código é colocado.
\end{itemize}

As seções a seguir resumem os requisitos básicos de formatação. Somente o básico da formatação e sintaxe do assembler serão apresentadas. Para informações adicionais, consulte o
manual de referência \textbf{yasm} (como observado no Capítulo \ref{cap1}, Introdução).

\section{Comentários}
O ponto-e-vírgula (\textbf{;}) é usado para anotar os comentários do programa. Comentários (usando o \textbf{;}) podem ser
colocados em qualquer lugar, inclusive após uma instrução. Qualquer caractere após o \textbf{;} é ignorado pelo montador. Isso pode ser usado para explicar as etapas executadas no código ou para comentar
seções de código.

\section{Valores numéricos}
Os valores numéricos podem ser especificados em decimal, hexadecimal ou octal. 

Ao especificar valores hexadecimais ou de base 16, eles são precedidos por um \textbf{0x}. Por exemplo, para especificar \textbf{127} como hexadecimal, usamos \textbf{0x7f}.

Ao especificar valores octais, ou base 8, eles são seguidos por um \textbf{q}. Por exemplo, para especificar \textbf{511} como octal, seria \textbf{777q}.

O radical padrão (base) é decimal, portanto, nenhuma notação especial é necessária para números decimais (base 10).

\section{Definindo constantes}
Constantes são definidas com \textbf{equ}. O formato geral é:

\begin{verbatim}
<nome> equ <valor>
\end{verbatim}

O valor de uma constante não pode ser alterado durante a execução do programa.

As constantes são substituídas pelos valores definidos durante o processo de montagem. Sendo assim, uma constante não recebe um local de memória. Isso torna a constante mais flexível, pois não é atribuído um tipo/tamanho específico (byte, palavra, palavra dupla, etc.). Os valores estão sujeitos às limitações de alcance do uso pretendido. Por exemplo, a seguinte constante,
\begin{verbatim}
TAMANHO equ 10000
\end{verbatim}
pode ser usada como uma word ou word dupla, mas não como um byte.

\section{Seção de dados}
Os dados inicializados devem ser declarados na seção ``\textit{section .data}''. Deve haver um espaço após a palavra `\textit{section}'. Todas as variáveis e constantes são colocadas nesta seção. Os nomes de variáveis devem começar com uma letra, seguida de letras ou números, incluindo alguns caracteres especiais (como o sublinha, ``\_''). Definições de variáveis devem incluir o nome, o tipo de dados e o valor inicial da variável.

O formato geral é:
\begin{verbatim}
<nomeVariável> <tipoDeDado> <valorInicial>
\end{verbatim}

Consulte as seções a seguir para uma série de exemplos usando vários tipos de dados.

Os tipos de dados suportados estão na Tabela \ref{tab:tiposDeDados}.

% Please add the following required packages to your document preamble:
% \usepackage{booktabs}
% \usepackage[table,xcdraw]{xcolor}
% If you use beamer only pass "xcolor=table" option, i.e. \documentclass[xcolor=table]{beamer}
\begin{table}[]
	\centering
	\begin{tabular}{|c|l|}
		\hline
		\rowcolor[HTML]{C0C0C0} 
		{\color[HTML]{000000} } Declaração & {\color[HTML]{000000} } \\ \hline
		\textbf{db}& variáveis de 8 bits \\ \hline
		\textbf{dw} & variáveis de 16 bits\\ \hline
		\textbf{dd} & variáveis de 32 bits\\ \hline
		\textbf{dq} & variáveis de 64 bits\\ \hline
		\textbf{ddq} & variáveis de 128 bits $\rightarrow$ inteira\\ \hline
		\textbf{dt} & variáveis de 128 bits $\rightarrow$ ponto flutuante\\ \hline
	\end{tabular}
\caption{Tipos de dados.}
\label{tab:tiposDeDados}
\end{table}

Estas são as diretivas principais do assembler para declarações de dados inicializados. Outras diretivas são referenciadas em diferentes seções.

Vetores inicializadas são definidos com valores separados por vírgula.

Alguns exemplos simples incluem:

\begin{verbatim}
bVar    db  10              ; variável byte
cVar    db  "H"             ; caracter único
strng   db  "Hello World"   ; string
wVar    dw 	5000            ; variável word
dVar    dd 	50000           ; variável de 32 bits
arr     dd 	100, 200, 300   ; array com 3 elementos
flt1    dd 	3.14159         ; ponto flutuante de 32 bits
qVar    dq 	1000000000      ; variável de 64 bits
\end{verbatim}

O valor especificado deve poder caber no tipo de dados especificado. Por exemplo, se o valor de uma variável de tamanho de byte é definido como 500, geraria um erro do assembler.

\section{Seção BSS}
Dados não inicializados são declarados na seção ``section .bss'' (Block Started by Symbol). Deve haver um espaço depois da palavra 'section'. Todas as 
variáveis não inicializadas são declaradas nesta seção. Os nome de variável começam com uma letra seguida de letras ou números, incluindo alguns caracteres especiais (como o sublinha, ``\_''). As definições de variáveis devem incluir o tipo de dados e a contagem.

O formato geral é:
\begin{verbatim}
<nomeDeVariável> <resTipo> <contador>
\end{verbatim}

Consulte as seções a seguir para uma série de exemplos usando vários tipos de dados.

Os tipos de dados suportados estão na Tabela \ref{tabDadosNInicializados}:

{\centering
\begin{table}[ht]
	\centering
	\begin{tabular}{|c|l|}
		\hline
		\rowcolor[HTML]{C0C0C0} 
		{\color[HTML]{000000} } Declaração & {\color[HTML]{000000} } \\ \hline
		\textbf{resb}& variáveis de 8 bits \\ \hline
		\textbf{resw} & variáveis de 16 bits\\ \hline
		\textbf{resd} & variáveis de 32 bits\\ \hline
		\textbf{resq} & variáveis de 64 bits\\ \hline
		\textbf{resdq} & variáveis de 128 bits \\ \hline
	\end{tabular}
    \caption{Declarações de variáveis não inicializadas}
    \label{tabDadosNInicializados}
\end{table}
} 

Essas são as diretivas do assembler principal para declarações de dados não inicializados. Outras diretivas são referenciadas em diferentes seções.

Alguns exemplos simples incluem:

\begin{verbatim}
bArr  resb 10  ; vetor com 10 elementos com tamanho byte
wArr  resw 50  ; vetor com 50 elementos com tamanho word
dArr  resd 100 ; vetor com 100 elementos com tamanho double
qArr  resq 200 ; vetor com 10 elementos com tamanho quad
\end{verbatim}

A matriz alocada não é inicializada para nenhum valor específico.

\section{Seção de texto}
O código é colocado na seção ``section .text''. Deve haver um espaço após a palavra 'section'. As instruções são especificadas uma por linha e cada uma deve ser uma instrução válida com os operandos necessários apropriados.

A seção de texto incluirá alguns cabeçalhos ou etiquetas que definem o ponto de entrada inicial do programa. Por exemplo, assumindo um programa básico usando o \textit{linker} padrão do sistema, as seguintes declarações devem ser incluídas.
\begin{verbatim}
global _start
_start:
\end{verbatim}

Nenhuma etiqueta ou diretiva especial é necessária para encerrar o programa. No entanto, um serviço do sistema deve ser usado para informar ao sistema operacional que o programa deve ser encerrado e os recursos, como memória, recuperados e reutilizados. Consulte o
exemplo de programa na seção a seguir.

\section{Programa de exemplo}
Um programa em linguagem assembly muito simples é apresentado para demonstrar a formatação do programa.

\begin{verbatim}
; Exemplo simples que demonstra o formato e o layout básicos do programa.

; Ed Jorgensen
; July 18, 2014

; ************************************************************
; Algumas declarações básicas de dados

section .data

; -----
; Define constantes

EXIT_SUCCESS      equ 0  ; operação bem sucedida
SYS_exit          equ 60 ; chamada de código para terminar

; -----
; Declarações de variáveis Byte (8-bits)

bVar1             db 17 
bVar2             db 9
bResult           db 0

; -----
; Declarações de variáveis word (16-bits)
wVar1             dw 17000
wVar2             dw 9000
wResult           dw 0

; -----
; Declarações de variáveis word dupla (32-bits)
dVar1             dd 17000000
dVar2             dd 9000000
dResult           dd 0

; -----
; Declarações de variáveis quad-word dupla (64-bits)
qVar1             dq 170000000
qVar2             dq 90000000
qResult           dq 0

; ************************************************************
; Seção de código
section .text
global _start
_start:

; ----------
; Executa uma série de operações de adição muito básicas
; para demonstrar o formato básico do programa.
; ----------
; Exemplo de byte
; bResult = bVar1 + bVar2
    mov    al, byte [bVar1]
    add    al, byte [bVar2]
    mov    byte [bResult], al

; ----------
; Exemplo word
; wResult = wVar1 + wVar2
    mov   ax, word [wVar1]
    add   ax, word [wVar2]
    mov   word [wResult], ax

; ----------
;  double-word
; dResult = dVar1 + dVar2
    mov   eax, dword [dVar1]
    add   eax, dword [dVar2]
    mov   dword [dResult], eax

; ----------
; Exemplo quad-word
; qResult = qVar1 + qVar2
    mov   rax, qword [qVar1]
    add   rax, qword [qVar2]
    mov   qword [qResult], rax

; ************************************************************
; Feito, termina o programa.
last:
    mov rax, SYS_exit
    mov rdi, EXIT_SUCCESS
    syscall
\end{verbatim}

Este exemplo de programa será referenciado e melhor explicado nos capítulos seguintes.

\section{Exercícios}
Abaixo estão algumas questões baseadas neste capítulo.

\subsection{Perguntas}
Abaixo estão algumas perguntas do questionário.
\begin{enumerate}
	\item Qual é o nome do montador sendo usado neste capítulo?
	\item Como os comentários são marcados em um programa em linguagem assembly?
	\item Qual é o nome da seção em que os dados inicializados foram declarados?
	\item Qual é o nome da seção em que os dados não inicializados foram declarados?
	\item Qual é o nome da seção onde o código é colocado?
	\item Qual é a declaração de dados para cada uma das seguintes variáveis com os valores fornecidos:
	\begin{enumerate}
		\item  variável de tamanho byte \textbf{\textit{bNum}} definida como $10_{10}$
		\item variável de tamanho de palavra \textbf{\textit{wNum}} definida como $10.291_{10}$
		\item variável \textbf{\textit{dwNum}} de tamanho de palavra dupla definida como $2.126.010_{10}$
		\item variável de tamanho quádruplo \textbf{\textit{qwNum} } definida como $10.000.000.000_{10}$
	\end{enumerate}
    \item Qual é a declaração de dados não inicializados para cada um dos seguintes casos:    
    \begin{enumerate}
    	\item matriz com tamanho de bytes chamada \textbf{\textit{bArr}} com 100 elementos
    	\item array com tamanho de palavra chamado \textbf{\textit{wArr} }com 3000 elementos
    	\item array com tamanho double-word, chamado \textbf{\textit{dwArr}}, com 200 elementos
    	\item array com tamanho de quad-word chamado \textbf{\textit{qArr}} com 5000 elementos
    \end{enumerate}
    \item Quais são as declarações necessárias para indicar o início de um programa (na seção texto)?

\end{enumerate}

